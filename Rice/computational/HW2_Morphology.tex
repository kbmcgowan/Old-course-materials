\documentclass[11pt]{article}
\usepackage{tipa}
\usepackage{tabularx}
\usepackage{booktabs}
\usepackage{multirow}
\usepackage{hyperref}
\usepackage[hmargin=1 in, margin=.75 in,nohead,nofoot]{geometry}

\begin{document}
\noindent
\begin{tabularx}{\textwidth}{X>{\raggedleft}X}
       \noindent Linguistics 409 & Homework 2\tabularnewline
       \noindent \emph{FSTs, Morphology, and Edit Distance} & Due: February 7, 2013\\
\end{tabularx}\\

\section{Finite State Transducers \& Morphology (30 pts)}

\subsection{(25 pts)} Please do problem 3.2 (p.81) and use the \texttt{dot} language (available on the CLEAR machines or for free download from \url{http://www.graphviz.org/}) to draw the resulting FST.  Full dot documentation is available from \url{http://www.graphviz.org/Documentation.php}.  Here is the dot input used to generate a figure like J\&M's 3.17:
     
     \begin{verbatim}
/*  Save to FST3-17.dot and generate image with command:
    dot -T png -o FST.png FST3-17.dot */

digraph jm_three_seventeen {
    /* tries to flatten the graph out, left to right */
    rankdir = LR;

    /* next line sets the shape for accept nodes */
    node [shape = doublecircle, style=filled, color=azure3]; q0 q1 q2;

    /* make all nodes the same shape and color */
    node [shape = circle, style=filled, color=azure3];

    /* define the edges of our graph */
    q0 -> q0 [ label = "^:&epsilon;\nother\n#", dir = back];
    q0 -> q1 [ label = "z,s,x"];

    q1 -> q0 [ label = "#,other"];
    q1 -> q1 [ label = "z,s,x", dir = back];
    q1 -> q2 [ label = "^:&epsilon;"];

    q2 -> q0 [ label = "#,other"];
    q2 -> q1 [ label = "z,x"];
    q2 -> q3 [ label = "&epsilon;:e"];
    q2 -> q5 [ label = "s"];

    q3 -> q4 [ label = "s"];

    q4 -> q0 [ label = "#"];

    q5 -> q0 [ label = "other"];
    q5 -> q1 [ label = "z,s,x"];
    q5 -> q2 [ label = "^:&epsilon;"];
}
     \end{verbatim}
     
     \textbf{Protip:} On the CLEAR machines: create a UNIX directory with \texttt{mkdir -p \~{}/Public/www/hw2/}, use \texttt{mv} or \texttt{cp} to put your png file in that directory, and then point your web browser at http://netid.web.rice.edu/hw2/ (where `netid' is your Rice NetID name) to easily see the image.

\subsection{(5pts)} Please answer J\&M problem 3.9:  Why does our figure 3.17 include a z,s,x arc from q5 to q1?

\section{Minimum Edit Distance: (30 pts)}

Please do problem 3.10.  Be sure to draw and hand in the matrices for \texttt{drive} $\rightarrow$ \texttt{brief} and \texttt{drive} $\rightarrow$ \texttt{divers}, provide edit distances between each pair of strings, explain which pair is closer, and highlight the best path through each matrix.


\end{document}
