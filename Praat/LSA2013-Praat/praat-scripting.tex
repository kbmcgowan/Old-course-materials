\documentclass [12pt]{article}
\usepackage{hyperref}
\usepackage{setspace,amssymb,latexsym,amsmath,amscd,epsfig,amsthm,wasysym}
\usepackage[margin=1in]{geometry}

\begin{document}
\clearpage\thispagestyle{empty}
   \begin{center}

     { \huge \textbf{Praat Scripting}}\\
     {\large LI 545 -- Summer 2013\\ M/W 1:30 - 3:20 in 2353 Mason Hall }
     \\--or--\\
     {\large T/Th 11:00 - 12:50 in  2001-B MLB } \\      
     \ \\

     \begin{tabular}{ l r }

       \begin{tabular}{l l}
          Instructor:& Kevin McGowan         \\
                            & kbmcgowan@stanford.edu    \\
                            & 2462 Mason Hall               \\
                            & \\
 %                           & \\                            
 %         TA: & Benjamin Chauvette \\
 %            & bdc3@rice.edu \\
 %            & 127 Herring Hall \\
       \end{tabular}

       &

       \begin{tabular}{l l }
          Office Hours: & 3:30 - 4:30 Monday\\
                           & 1:00 - 2:00 Thursday\\    
                                   & or by appointment\\         
                                   \\
 %                                  \\
 %                                  Office Hours: & Th 3:00 - 4:00 pm\\                                
 %                                  \\
 %                                  \\      
       \end{tabular}
     \end{tabular}
   \end{center}
   \ \\

\section*{Course Description and Goals}

This course introduces basic automation and scripting skills for linguists using Praat. The course will expand upon a basic familiarity with Praat and explore how scripting can help you automate mundane tasks, ensure consistency in your analyses, and provide implicit (and richly-detailed) methodological documentation of your research.  Our main goals will be:

     \begin{enumerate}
    \item To expand upon a basic familiarity with Praat by exploring the software's capabilities and learning the details of its scripting language.
    \item To practice a set of scripting best practices to help you not only write and maintain your own scripts but evaluate scripts written by others.
     \end{enumerate}

The course assumes participants have read and practiced with the Intro from Praat's help manual. You should go do that now if you haven't already.  Go on.  I'll wait.
    
 \section*{Requirements}
  \begin{enumerate}
       \item \textbf{Attendance \& Participation} \hfill{} 60\%\\
            Reading will be minimal for this course, but hands-on practice is everything. Come, listen, participate, and ask questions.
       \item \textbf{Homework Assignments} \hfill{} 40\%\\
	    Along with numerous (ungraded) in-class exercises, there will be 4 homework assignments to hand-in. These will be graded on a scale from `exists' to `does not exist'. Assignments that fail to exist will receive a corresponding amount of credit.
   \end{enumerate}

Our minimal required readings are from \underline{\href{http://www.fon.hum.uva.nl/praat/manual/Scripting.html}{the Praat manual}}, are listed in bold face on the course schedule, and should ideally be completed prior to the date listed on the syllabus.  I recommend the following additional resources:

\begin{description}
     \item[\href{http://savethevowels.org/praat/UsingPraatforLinguisticResearchLatest.pdf}{Will Styler}] From the 2011 LSA Praat workshop:\\\href{http://savethevowels.org/praat/UsingPraatforLinguisticResearchLatest.pdf}{http://savethevowels.org/praat/UsingPraatforLinguisticResearchLatest.pdf}
     \item[\href{http://www.fon.hum.uva.nl/david/sspbook/sspbook.pdf}{David Weenink}] A draft book by one of Praat's main authors.  Chapter 4 covers scripting. \href{http://www.fon.hum.uva.nl/david/sspbook/sspbook.pdf}{http://www.fon.hum.uva.nl/david/sspbook/sspbook.pdf}
\end{description}

\section*{Course Schedule}

\subsection*{Day 1: Hello Praat, Hello world!}
We'll briefly review key concepts in Praat (the various windows, dynamic versus fixed menus, object properties, etc.) before tackling the classic `Hello, world!' program in (at least) four different ways.\\

Reading: Intro 1-3, Scripting 1

\subsection*{Day 2: Working with data and objects}
An introduction to Praat's internal data types, with a heavy emphasis on strings.\\

Reading: Scripting 2-4

\subsection*{Day 3: Annotation, Labeling, and Segmentation}
Textgrids are Praat's way of allowing you to make your segmentation and analysis repeatable --and they offer many opportunities to simplify your life via scripting.\\

Reading: Praat Intro 7

\subsection*{Day 4: Getting more complicated: Variables, Flow Control, and Conditionals}
Reading: Praat Scripting 5

\subsection*{Day 5: Even more complicated: Variables, Flow Control, and Conditionals}


\subsection*{Day 6: Blurring the line between scripting and programming}
Reading: Scripting 6-7

\subsection*{Day 8: Using old code}
Finding and using scripts from the web: variable-substitution-free scripting, checking others' math, the dangers of the `Remove' command, etc.


\end{document}
